%----------------------------------------------------------------------------------------
%	PACKAGES AND DOCUMENT CONFIGURATIONS
%----------------------------------------------------------------------------------------
\documentclass[11pt]{article}

\setlength{\textheight}{9truein}
\setlength{\topmargin}{-0.9truein}
\setlength{\parindent}{0pt}
\setlength{\parskip}{10pt}
\setlength{\columnsep}{.4in}
\newcommand{\beq}{\begin{equation}}
	\newcommand{\eeq}{\end{equation}}
\renewcommand{\abstractname}{}

\usepackage{graphicx}
\usepackage{url}
\usepackage{mathtools}

%----------------------------------------------------------------------------------------
%	REPORT INFORMATION
%----------------------------------------------------------------------------------------
\title{Pixel space convolution using sparse matrices} % Report title
\author{Pedro \textsc{Flux\'a}} % Author name(s), add additional authors like: '\& James \textsc{Smith}'
\date{\today} % Date of the report



\begin{document}
\pagestyle{plain}
\maketitle
\setlength{\parindent}{0pt}
%----------------------------------------------------------------------------------------
%	Abstract
%----------------------------------------------------------------------------------------
\begin{abstract} No abstract for now.
	\vspace{.3in} 
\end{abstract}

\section{Objective}
	
	Propose a novel way of calculating full-sky convolution of a polarized sky with antenna beams in pixel space.
	
\section*{Note}
	
This section uses the same notation as the paper that describes PISCO.
	
\section{Beams of a Polarization Sensitive Bolometer}
	
A Polarization Sensitive Bolometer (PSB) is a detection device that uses a pair of bolometers to sense both the total intensity (Stokes parameter $I$) and the degree of linear polarization in incoming light (Stokes parameter $Q$ and $U$). Denoting each bolometer of the PSB as $a$ and $b$ we can write \footnote{cite Rossett et al and link the calculations that show this formulation is equivalent to O'Dea et al.}
	
\begin{eqnarray}
	\tilde{I}_{a} &=& |\vec{E}_{a}|^2 \\
	\tilde{I}_{b} &=& |\vec{E}_{b}|^2 \\
	\tilde{Q}_{a} &=& |\vec{E}_{a \parallel}|^2 - |\vec{E}_{a \times}|^2 \\
	\tilde{Q}_{b} &=& |\vec{E}_{b \parallel}|^2 - |\vec{E}_{b \times}|^2 \\
	\tilde{U}_{a} &=& 2 \Re{\left( \vec{E}_{a \parallel} \vec{E}_{a \times} \right)} \\
	\tilde{U}_{b} &=& 2 \Re{\left( \vec{E}_{b \parallel} \vec{E}_{b \times} \right)} \\
	\tilde{V}_{a} &=& -2 \Im{\left( \vec{E}_{a \parallel} \vec{E}_{a \times} \right)} \\
	\tilde{V}_{b} &=& -2 \Im{\left( \vec{E}_{b \parallel} \vec{E}_{b \times} \right)} \quad ,
\end{eqnarray}

\noindent
where $\vec{E}$ denotes the time-averaged electric field density at co-latitude $\rho$ and longitude $\sigma$ (see \ref{fig:coordinate_systems}). 
Subscript $\parallel$ is used to denote the co-polarized field, while $\times$ is used for the cross-polarized field. 

\textsl{Maybe add a note about polarization efficiency}

\section{Sky model}

The sky can be modeled as a set of 4 scalar fields

\begin{equation}
	S^i = S^i(\theta, \phi) = (I, Q, U, V) \quad ,
\end{equation}

\noindent
where $I, Q, U$ and $V$ are Stokes parameters\footnote{Note about definition of polarization} of light coming from co-latitude $\theta$
and longitude $\phi$ in the sky basis. 

\section{Coupling of sky and PSB beams in pixel space}

Note that if the PSB beams and sky are considered discrete quantities, then each pixel $k$ will have an associated pair of coordinates in its corresponding basis. 
In pixel space, the power (per unit-wavelength) measured by a PSB pointing at $\bar{q}_0$ becomes

\begin{equation}
\label{eq:coupling}
	d_{\alpha} = \sum^{N_s}_{k=1} \prescript{}{k}{ D_{\alpha,i}(\chi, \epsilon, s_{\alpha})} \prescript{}{k}{S^{i}} \quad ,
\end{equation}
 
\noindent
where $\alpha$ denotes $a$ and $b$.

The quantity $D$ for detector $a$ is 
\begin{equation}
\begin{aligned}
D_{a, i}(\chi, \epsilon, s_a) &=& \frac{s_a}{2} \big{[} \\
 \tilde{I}_{a} + \epsilon \tilde{I}_{b} , \\
\left(\tilde{Q}_{a} - \epsilon \tilde{Q}_{b} \right) \cos(2 \chi) - \left(\tilde{U}_{a} - \epsilon \tilde{U}_{b} \right) \sin(2 \chi) , \\
\left(\tilde{U}_{a} - \epsilon \tilde{U}_{b} \right) \cos(2 \chi) + \left(\tilde{Q}_{a} - \epsilon \tilde{Q}_{b} \right) \sin(2 \chi) , \\
0 \big{]} & \quad ,
\end{aligned}
\end{equation}

\noindent
and the quantity $D$ for detector $b$
\begin{equation}
	\begin{aligned}
		D_{b, i}(\chi, \epsilon, s_b) &=& \frac{s_b}{2} \big{[} \\
		\tilde{I}_{b} + \epsilon \tilde{I}_{a} , \\
		-\left(\tilde{Q}_{b} - \epsilon \tilde{Q}_{a} \right) \cos(2 \chi) + \left(\tilde{U}_{b} - \epsilon \tilde{U}_{a} \right) \sin(2 \chi) , \\
		-\left(\tilde{U}_{b} - \epsilon \tilde{U}_{a} \right) \cos(2 \chi) - \left(\tilde{Q}_{b} - \epsilon \tilde{Q}_{a} \right) \sin(2 \chi) , \\
		0 \big{]} & \quad ,
	\end{aligned}
\end{equation}

\section{Computation of $D$}

In practice, discrete quantities such as sky and PSB beams are represented as vectors. Note that the definition of $D$ is implicitly assuming that 
the PSB beam is aligned with the sky: more precisely, that pixel $k$ of the beam corresponds to pixel $k$ of the sky. In this sense $D$ hides a \textsl{rotation}
of the PSB beams from the beam basis to the sky basis. 

Calculating the \textsl{rotated} PSB beam can be represented as a matrix-vector product. For each component ($i$) of the beam, its rotated counterpart becomes

\begin{equation}
	\tilde{X}^{'} = \mathbf{R} \tilde{X}^{T} \quad ,
\end{equation}

\noindent
where $\mathbf{R}$ is an $N_s \times N_b$ sparse matrix with exactly 4 non-zero entries per row\footnote{For bi-linear interpolation}.

In some way, every row of $\mathbf{R}$ represents the dot-product of 4 pixels from the PSB beam component (North, South, East and West pixels around $\bar{q}_{0}$) 
and their interpolation weights. Denoting $S$, $N$, $E$ and $W$ (all integers from $0$ to $N_b$) to neighbor pixels and $z_N$, $z_S$, $z_E$ and $z_W$ to
the interpolating weights

\begin{equation}
	\prescript{}{k}{ \tilde{X}^{'} } = \prescript{}{k}{ \mathbf{R} } \tilde{X}^{T} = \big{[} 0 \cdots z_N \cdots z_S \cdots z_E \cdots z_W \cdots 0 \big{]} \cdot \tilde{X}^{T}
\end{equation}

\section{Remarks of algorithm}

\begin{itemize}
	\item Formalism is designed for full-sky convolution.
	\item Every row of $\mathbf{R}$ can be calculated in parallel using already existing libraries (like HEALPix) 
	\item Matrix $\mathbf{R}$ is very sparse, and can be represented efficiently using compression techniques (like CSR)
	\item The (sparse) matrix-vector product can be calculated in accelerators, like a GPU, using existing libraries for Sparse Linear Algebra (like cuSparse from NVIDIA).
\end{itemize}
%
%\begin{figure}[!h!t]
%	\centering
%	\includegraphics[width=\columnwidth]{Figures/placeholder.jpg}
%	\caption{Figure of the coordinate systems.}
%	\label{fig:coordinate_systems}
%\end{figure} 
	
	
	\section*{References}
	
	\url{http://www.xkcd.com/242/}
	
\end{document}